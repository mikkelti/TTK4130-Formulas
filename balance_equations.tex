\part{Balance equations}
\section{Kinematics of flow}
\subsection{Material derivative}
Let $\mathbf{x}$ be the position of some fluid particle, with velocity $\dot{\mathbf{x}} = \mathbf{v}$. Further, let $\varphi$ be some quantity related to the particle, e.g. its temperature. The material derivative of $\varphi$ is then defined as
\begin{equation}
    \frac{D\varphi}{Dt} = \frac{\partial \varphi}{\partial t} + \mathbf{v}^\top \nabla \varphi
\end{equation}

\section{Mass, momentum and energy balances}
\subsection{Mass balance}
Level of tank
\begin{align}
    \frac{d}{dt}(\rho A h) = w_1 - w_2\\
    \dot{h} = \frac{1}{\rho A} (w1 - w2)
\end{align}
Remember that the pressure at the bottom of a (open) tank is $\rho g h$.

\subsection{Momentum balance}
Bernoulli's equation for stationary frictionless flow along a streamline for an incompressible fluid
\begin{equation}
    \frac{1}{2} (v_2^2 - v_1^2) + \frac{p_2 - p_1}{\rho} + (z_2 - z_1)g = 0
\end{equation}
See page 426-427 for relevant examples on momentum balances.

\subsection{Energy balance}
The total energy of a volume element $dV$ is
\begin{equation}
    \rho e dV,
\end{equation}
where $e = u + \frac{1}{2}v^2 + \phi$, i.e. the sum of specific internal, kinetic and potential energy.
Internal energy, enthalpy, heat capacities and temperature have the following relationships
\begin{align}
    h = c_p T\\
    u = c_v T
\end{align}
Note that all quantities (except temperature) are specific (per unit mass) in these equations. The specific enthalpy is given by
\begin{equation}
    h = u + \frac{p}{\rho},
\end{equation}
where $u$ is the internal energy. See page 443-445 for relevant examples of energy balances.