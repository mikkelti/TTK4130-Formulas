\part{Dynamics}
\section{Rigid body kinematics}
\subsection{Rotation matrices}
Let $\{\vec{a}_1, \vec{a}_2, \vec{a}_3\}$ and $\{\vec{b}_1, \vec{b}_2, \vec{b}_3\}$ be the orthogonal bases of two coordinate frames. Then the coordinate transformation from frame $b$ to frame $a$ is given by
\begin{equation}
    \mathbf{R}^a_b = 
    \begin{pmatrix}
    \mathbf{b}^a_1 & \mathbf{b}^a_2 & \mathbf{b}^a_3 
    \end{pmatrix}
\end{equation}
That is,
\begin{equation}
    \mathbf{v}^a = \mathbf{R}^a_b \mathbf{v}^b
\end{equation}
The $\mathbf{R}^a_b$ matrix can also be thought of simply as a rotation of a vector. For a second, forget that the $b$ frame exists, and think only about the $a$ frame. A simple example is 
\begin{equation}
    \mathbf{b}^a_1 = \mathbf{R}^a_b \mathbf{a}^a_1
\end{equation}
which shows that the first basis vector of the $a$ frame is rotated to the first basis vector of the $b$ frame, when everything is referred to frame $a$. In this sense the matrix represents a rotation from $a$ to $b$.

\subsection{Homogeneous transformation matrices}
\begin{equation}
    T^a_b = 
    \begin{pmatrix}
        R^a_b & r^a_{ab}\\
        \mathbf{0}^\top & 1
    \end{pmatrix}
\end{equation}
where $r^a_{ab}$ is the origin of frame $b$ in $a$ coordinates.

\subsection{Differentiation of vectors and matrices}
\begin{equation}
    \frac{\prescript{a}{}d}{dt} \vec{u} = \frac{\prescript{b}{}d}{dt} \vec{u} + \vec{\omega}_{ab} \times \vec{u},
\end{equation}
Skew-symmetric form of coordinate vector
\begin{equation}
    \mathbf{u}^\times := 
    \begin{pmatrix}
    0 & -u_3 & u_2\\
    u_3 & 0 & -u_1\\
    -u_2 & u_1 & 0
    \end{pmatrix}
\end{equation}
Skew-symmetric form of angular velocity vector
\begin{equation}
    (\boldsymbol{\omega}^a_{ab})^\times = \dot{\mathbf{R}}^a_b (\mathbf{R}^a_b)^\top
\end{equation}
Coordinate transformation
\begin{equation}
    (\boldsymbol{\omega}^a_{ab})^\times = \mathbf{R}^a_b (\boldsymbol{\omega}^b_{ab})^\times \mathbf{R}^b_a, \quad
    \mathbf{R}^b_a = (\mathbf{R}^a_b)^{-1} = (\mathbf{R}^a_b)^\top
\end{equation}

\subsection{Kinematic differential equations}
For Euler angles, when the middle rotation is $\frac{\pi}{2}$ radians, the $E$ matrix in the kinematic differential equation is singular. This is because this rotation moves the third axis of rotation to the first axis of rotation, such that we lose a degree of freedom. 

\subsection{Coordinate systems}
Cylindrical coordinates $(r, \theta, z)$:
\begin{align}
    x &= r\cos(\theta)\\
    y &= r\sin(\theta)\\
    z &= z\\
    r &= \sqrt{x^2 + y^2}\\
    \theta &= 
    \begin{cases}
    0 \text{ if } x = 0 \text{ and } y = 0\\
    \arctan(\frac{y}{x}) \text{ if } x > 0\\
    \arctan(\frac{y}{x}) + \pi \text{ if } x < 0 \text{ and } y \geq 0\\
    \arctan(\frac{y}{x}) - \pi \text{ if } x < 0 \text{ and } y < 0\\
    \frac{\pi}{2} \text{ if } x = 0 \text{ and } y > 0\\
    -\frac{\pi}{2} \text{ if } x = 0 \text{ and } y < 0
    \end{cases}\\
    dV &= dxdydz = rdrd\theta dz
\end{align}
Spherical coordinates $(r, \varphi, \theta)$, $\varphi$ angle from $z$-axis:
\begin{align}
    x &= r\sin(\varphi)\cos(\theta)\\
    y &= r\sin(\varphi)\sin(\theta)\\
    z &= r\cos(\varphi)\\
    r &= \sqrt{x^2 + y^2 + z^2}\\
    \theta &\text{ defined as above.}\\
    dV &= dxdydz = r^2\sin(\varphi)dr d\varphi d\theta
\end{align}
Note that $r \geq 0$, $0 \leq \theta \leq 2\pi$ and $0 \leq \varphi \leq \pi$ (typical definition). The definition of $\theta$ above has the range $(-\pi, \pi]$, to obtain only positive results $2\pi$ can be added to negative values.

\subsection{The center of mass}
The mass of a rigid body $b$ is
\begin{equation}
    m = \int_{b}dm = \int_{b} \rho(x,y,z)dV
\end{equation}
The center of mass $\vec{r}_c$ is defined as
\begin{equation}
    \vec{r}_c = \frac{1}{m} \int_{b} \vec{r}_p dm
\end{equation}
where $\vec{r}_p$ is the position of a mass element $dm$ that is fixed in frame $b$. The $x$--coordinate of the center of mass is given by
\begin{align}
    x_c = \frac{1}{m} \iiint \limits_b x_p \rho(x,y,z)dV
\end{align}
The definitions for $y$ and $z$ are exactly the same. Typically $(x_p, y_p, z_p) = (x, y, z)$.

\subsection{Other useful formulas}
Relation between linear and angular velocity
\begin{equation}
    v = \omega r
\end{equation}

\section{Newton-Euler equations of motion}
\subsection{Kinetic energy}
\begin{equation}
    \mathcal{T} = \frac{1}{2} m (\mathbf{v}_c^b)^\top \mathbf{v}_c^b +  \frac{1}{2} (\boldsymbol{\omega}_{ib}^b)^\top \mathbf{M}_{b/c}^b \boldsymbol{\omega}_{ib}^b
\end{equation}
The subscript $c$ denotes the center of mass and the superscript $b$ denotes a coordinate vector/matrix in frame $b$. $\boldsymbol{\omega}_{ib}^b$ is the angular velocity of frame $b$ relative to frame $i$. $\mathbf{M}_{b/c}^b$ is the inertia matrix of $b$ about $c$, i.e. the inertia matrix of the rigid body about the center of mass.

\subsection{Inertia matrix}
\begin{equation}
    \mathbf{M}_{b/c}^b = -\int_b (\mathbf{r}^b)^\times (\mathbf{r}^b)^\times dm = \int_b \left[ (\mathbf{r}^b)^\top \mathbf{r}^b \mathbf{I} - \mathbf{r}^b (\mathbf{r}^b)^\top \right] dm
\end{equation}
Fun facts: Swap the $b$ superscripts with $i$ on the right hand side to get $\mathbf{M}_{b/c}^i$. $\mathbf{M}_{b/c}^b$ is positive definite, since the kinetic energy $\mathcal{T} \geq 0$. Note that the integral above is a triple integral of a $3 \times 3$-matrix. About a specified axis the formula reduces to
\begin{equation}
    I = \int_{b} (\mathbf{r}^b)^\top \mathbf{r}^b dm
\end{equation}

\subsection{Parallel axis theorem}
The inertia matrix of $b$ about a point $o$ is given by
\begin{equation}
    \mathbf{M}_{b/o}^b = \mathbf{M}_{b/c}^b - m (\mathbf{r}_g^b)^\times (\mathbf{r}_g^b)^\times =
    \mathbf{M}_{b/c}^b + m\left[ (\mathbf{r}_g^b)^\top \mathbf{r}_g^b \mathbf{I} - \mathbf{r}_g^b (\mathbf{r}_g^b)^\top \right]
\end{equation}
where $\mathbf{r}_g^b$ is the vector from the point $o$ to the center of mass $c$. If $o$ is the origin this corresponds to $\mathbf{r}_c^b$. In its simplest form with two parallel axes, the formula reduces to
\begin{equation}
    I = I_c + md^2
\end{equation}
where $I_c$ is the moment of inertia about the axis through the center of mass and $d$ is the distance between the axes.

\subsection{Other useful formulas}
Relationships between torque, angular momentum and angular velocity
\begin{align}
    \vec{\tau} = \vec{r} \times \vec{F}\\
    \vec{\tau} = \frac{d\vec{L}}{dt}\\
    \vec{\tau} = I \dot{\omega}\\
    P = \tau \omega
\end{align}

\section{Lagrangian dynamics}
Lagrangian
\begin{equation}
    \mathcal{L}(\mathbf{q, \dot{q}}, t) = \mathcal{T}(\mathbf{q, \dot{q}}, t) - \mathcal{U}(\mathbf{q})
\end{equation}
Lagrange's equation of motion
\begin{equation}
    \frac{d}{dt} \left( \frac{\partial \mathcal{L}}{\partial \dot{q}_i} \right) - \frac{\partial \mathcal{L}}{\partial q_i} = \tau_i
\end{equation}
Generalized force
\begin{equation}
    Q_i = \sum_{k=1}^N \frac{\partial \vec{r}_k}{\partial q_i} \cdot \vec{F}_k
\end{equation}
