\part{Modelica reference}
\todo[inline]{Potentially add some stuff from exercise 4, like user-defined types. Look at connectors.} Modelica is an object-oriented, equation-based modeling language. This section shows some example models. An important thing to note is that you need the same number of equations as the number of variables.
\begin{minted}{modelica}
    model Oscillator "Descriptive comment"
        constant Real m = 1; // This comment is ignored by the compiler.
        parameter Real c = 1, k = 1;
        Real x(start=1); // 'start' is a hint to the compiler.
        Real vx;
    equation
        der(x) = vx;
        m*der(vx) + c*vx + k*x = 0;
        // Could also add algebraic equations to make it a DAE.
    intial equation
        vx = 0; // This is an actual constraint, unlike 'start'.
    end Oscillator;
\end{minted}
Next up, we look at a model which inherits from \texttt{Oscillator}, by use of the keyword \texttt{extends}:
\begin{minted}{modelica}
    model DrivenOscillator
        extends Oscillator;
        Real F;
    equation
        F = sin(time);
        m*der(vx) + c*vx + k*x = F;
    end DrivenOscillator;
\end{minted}