\part{Modelica reference}
Modelica is an object-oriented, equation-based modeling language. This section shows some example models. An important thing to note is that you need the same number of equations as the number of variables.
\begin{minted}{modelica}
    model Oscillator "Descriptive comment"
        constant Real m = 1; // This comment is ignored by the compiler.
        parameter Real c = 1, k = 1;
        Real x(start=1); // 'start' is a hint to the compiler.
        Real vx;
    equation
        der(x) = vx;
        m*der(vx) + c*vx + k*x = 0;
        // Could also add algebraic equations to make it a DAE.
    intial equation
        vx = 0; // This is an actual constraint, unlike 'start'.
    end Oscillator;
\end{minted}
Next up, we look at a model which inherits from \texttt{Oscillator}, by use of the keyword \texttt{extends}:
\begin{minted}{modelica}
    model DrivenOscillator
        extends Oscillator;
        Real F;
    equation
        F = sin(time);
        m*der(vx) + c*vx + k*x = F;
    end DrivenOscillator;
\end{minted}
\newpage
The third example deals with the use of connectors.
\begin{minted}{modelica}
    connector Pin "Electrical pin";
        Modelica.SIunits.Voltage v; // Potential/energy level
        flow Modelica.SIunits.Current i; // Flow
    end Pin;

    partial model TwoPin // partial because of 6 var. and only 5 eq.
        Pin p, n;
        Modelica.SIunits.Voltage v; // Voltage across component
        Modelica.SIunits.Current i; // Current through component
    equation
        v = p.v - n.v;
        p.i + n.i = 0 // KCL
        i = p.i // Current into component
    end TwoPin;

    model Inductor
        extends TwoPin;
        parameter Modelica.SIunits.Inductance L;
    equation
        v = L * der(i);
    end Inductor;
\end{minted}
Declaring your own type (based on \texttt{Real}) in Modelica is done in the following way:
\begin{minted}{modelica}
    type Voltage = Real(unit = "V", min = 0, max = 5.0);
\end{minted}