\part{Simulation}
We are concerned with solving the IVP
\begin{equation}
    \Dot{y} = f(y,t), \quad y(t_0) = y_0
\end{equation}
The stability of a numerical method is ensured if $|R(h\lambda_i)| \leq 1$ for all eigenvalues $\lambda_i$.

\section{Stability functions}
\subsection{ERK methods}
\begin{equation}
    R_E(h\lambda) = \det \left[ I - h\lambda (A - \mathbf{1} b^\top) \right], \enskip \text{where} \enskip  \mathbf{1} = (1, ... , 1)^\top
\end{equation}
Note that $R_E(h\lambda)$ will be a polynomial in $h\lambda$ of order less than or equal to $\sigma$ (the number of stages).

\subsection{IRK methods}
\begin{align}
    R(h\lambda) = \left[ 1 + h\lambda b^\top (I-h\lambda A)^{-1} \mathbf{1} \right]\\
    R(h\lambda) = \frac{\det \left[ I - h\lambda(A - \mathbf{1}b^\top) \right]}{\det(I - h\lambda A)}
\end{align}

\subsection{Aliasing}
The \textit{Nyquist frequency} is half of the sampling rate
\begin{equation}
    \omega_{\text{Nyquist}} = \frac{1}{2} \cdot \frac{2\pi}{h}, \enskip \text{where $h$ is the step size.}
\end{equation}
Two systems oscillating at a low frequency $\omega < \omega_{\text{Nyquist}}$ and a high frequency $\omega + 2k\frac{\pi}{h} > \omega_{\text{Nyquist}}$ ($k$ integer) will intercept at all sampling points, and therefore a solver will not be able to distinguish them. More specifically, the solver will believe that the system with higher frequency is the system with lower frequency, when fitting the curve.

\subsection{A- and L-stability}
\textbf{Definition:} A method is A-stable if $|R(h\lambda)| \leq 1 \enskip \forall \enskip \operatorname{Re} \lambda \leq 0$.\\
This definitions means that an A-stable method is stable for all stable test systems $\dot{y} = \lambda y$. Note also that no ERK method can be A-stable, since $|R_E(h\lambda)| \to \infty$ as $|\lambda| \to \infty$.\\
\textbf{Definition:} A method is L-stable if it is A-stable and $|R(j\omega h)| \to 0$ when $\omega \to \infty \enskip \forall$ systems $\dot{y} = j\omega y$.\\
A-stable methods can suffer from aliasing for systems with fast dynamics (faster than Nyquist frequency), whereas an L-stable method will simply damp out these fast dynamics. This means that the L-stable method might give a better qualitative representation of what the actual solution looks like.

\subsection{Stiffly accurate methods and algebraic stability}
\textbf{Definition:} A method is stiffly accurate if
\begin{equation}
    \det(A) \neq 0 \text{ and } b = A^\top [0, 0, ..., 1]^\top
\end{equation}
Note: A-stable and stiffly accurate $\implies$ L-stable.\\
\textbf{Definition:} A method is algebraically stable if 
\begin{equation}
    M = \operatorname{diag}(b)A + (\operatorname{diag}(b)A)^\top + bb^\top
\end{equation}
is positive semi-definite. Note: Algebraically stable $\implies$ A-stable.

\subsection{DAEs}
A fully implicit ODE, $F(\dot{x}, x, u) = 0$ is a DAE if $\frac{\partial F}{\partial \dot{x}}$ is rank deficient.